%%%%%%%%%%%%%%%%%%%%%%%%%%%%%%%%%%%%%%%%%%%%%%%%%%%%%%%%%%%%%%%%%%%%%%
% LaTeX Example: Project Report
%
% Source: http://www.howtotex.com
%
% Feel free to distribute this example, but please keep the referral
% to howtotex.com
% Date: March 2011 
% 
%%%%%%%%%%%%%%%%%%%%%%%%%%%%%%%%%%%%%%%%%%%%%%%%%%%%%%%%%%%%%%%%%%%%%%

% Edit the title below to update the display in My Documents
%\title{Project Report}
%
%%% Preamble
\documentclass[paper=a4, fontsize=12pt]{article}
\usepackage[T1]{fontenc}
\usepackage{fourier}
\usepackage[utf8]{inputenc}
\usepackage[french]{babel}

% English language/hyphenation
\usepackage[protrusion=true,expansion=true]{microtype}	
\usepackage{amsmath,amsfonts,amsthm} % Math packages
\usepackage[pdftex]{graphicx}	
\usepackage{url}
\usepackage[bottom=10em]{geometry}
\usepackage{float}
\usepackage{xcolor}
\usepackage{enumitem}
\renewcommand\descriptionlabel[1]{\textbf{#1 :}}
\usepackage{pdfpages}
\usepackage{rotating}

%%% Custom sectioning
%\usepackage{sectsty}
%\allsectionsfont{\normalfont\scshape}

%% Language definition package (for XML Annexe)
\usepackage{listings}
\usepackage{color}

%% Local modification of margins
\newenvironment{changemargin}[2]{\begin{list}{}{%
      \setlength{\topsep}{0pt}%
      \setlength{\leftmargin}{0pt}%
      \setlength{\rightmargin}{0pt}%
      \setlength{\listparindent}{\parindent}%
      \setlength{\itemindent}{\parindent}%
      \setlength{\parsep}{0pt plus 1pt}%
      \addtolength{\leftmargin}{#1}%
      \addtolength{\rightmargin}{#2}%
    }\item }{\end{list}}
%%

%%% Custom headers/footers (fancyhdr package)
%\usepackage{fancyhdr}
%\pagestyle{fancyplain}
%\fancyhead{}											% No page header
%\fancyfoot[L]{}											% Empty 
%\fancyfoot[C]{}											% Empty
%\fancyfoot[C]{\thepage}									% Pagenumbering
%\renewcommand{\headrulewidth}{0pt}			% Remove header underlines
%\renewcommand{\footrulewidth}{0pt}				% Remove footer underlines
%\setlength{\headheight}{13.6pt}


%%% Equation and float numbering
\numberwithin{equation}{section}		% Equationnumbering: section.eq#
\numberwithin{figure}{section}			% Figurenumbering: section.fig#
\numberwithin{table}{section}				% Tablenumbering: section.tab#

%Graphics path
%\graphicspath{./Images/}

%%% Maketitle metadata
\newcommand{\horrule}[1]{\rule{\linewidth}{#1}} 	% Horizontal rule

\title{
  %\vspace{-1in} 			
  \usefont{OT1}{bch}{b}{n}
  \horrule{1.5pt} \\[0.5cm]	
  \Huge \textbf{Cahier des charges} \\ [10pt]
  \Huge Projet VHDL S7\\ [15pt]
  \LARGE Année scolaire 2015-2016 \\ 
  \horrule{1.5pt} \\[0.5cm]
  %
}

\author{
  \huge \underline{Encadrant} : \LARGE Sylvie RENAUD\\[20pt]
  \normalfont 							
  \huge \textbf{Binôme} : \Large Adrian BARBE - Charles DELCROIX \\[5pt]
		\normalsize
}
\date{}

%%% Begin document
\begin{document}
\maketitle
\newpage

\tableofcontents

\newpage

\section{Explication du projet}

Le but du projet est de transmettre un signal sonore, le signal à transmettre et le signal transmis n'étant pas le même.
Cela permet une transmission audible mais non compréhensible, voir une transmission inaudible en passant par de l'ultrason.
\\\\Le projet se décompose alors en 5 étapes.
\begin{itemize}[label=$\square$,leftmargin=* ,parsep=0cm,itemsep=0cm,topsep=0cm]
\item 1er étape: enregistrer un signal sonore sur un micro, lequel est intégré à la carte. Le signal de départ pourrat être à terme de la voix.
\item 2ème étape: translater en fréquence \textbf{l'ensemble} du signal. Le processus lié à cette étape est indépendant du type de signal acquit à l'étape 1.
\item 3ème étape: émettre sur le haut parleur (étape 3a) et au même moment capter grâce au micro (étape 3b) le signal modifé.
\item 4ème étape: translater en fréquence \textbf{l'ensemble} du signal.
\item 5ème étape: emettre le signal, dans l'idéal se dernier est alors identique à celui de l'étape 1. En réalité il sera sans doute quelque peu déformé.

\begin{figure}[h!]
\centerline{\includegraphics[width=20cm]{Illustrations/decoupage_etapes_1}}
\caption{\label{Illustrations/decoupage_etapes_1} Découpage en étapes 1}
\end{figure}

On remarque alors que les étapes 1 et 3b, 2 et 4, 3a et 5 sont similaires. Ce qui peut nous faire penser au deuxième schéma suivant:

\begin{figure}[h!]
\centerline{\includegraphics[width=20cm]{Illustrations/decoupage_etapes_2}}
\caption{\label{Illustrations/decoupage_etapes_2} Découpage en étapes 2}
\end{figure}


\end{itemize}

\section{Matériel nécessaire}

Dans l'idéal le projet nécessite deux cartes Nexys4, cependant nous sommes en mesure de le réaliser avec une seule. Sur cette carte nous allons utiliser le micro interne.
\textbf{Cependant nous allons avoir besoin d'un haut parleur, la carte n'en possédant pas en interne.}

\section{Dates butoirs}
Réalisation cahier des charges: 1ère séance

\section{Difficultés probables}
Mémoire nécessaire pour enregistrement initial. Calcul de durée nécessaire.

%%% End document
\end{document}